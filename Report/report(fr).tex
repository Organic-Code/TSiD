\documentclass[12pt,a4paper,twoside]{article}
%Packages de langue
	\usepackage[utf8]{inputenc}
	%\usepackage[latin1]{inputenc}
	\usepackage[T1]{fontenc}
	\usepackage[english,frenchb]{babel} %deuxième langue = langue principale

%Création d'un layout
	%\usepackage{layout}

%Modification des marges
	\usepackage[top=2cm, bottom=1.8cm, left=1.8cm, right=1.8cm, head=14pt, foot=36pt]{geometry}

%Interligne
	%\usepackage{setspace}

%Soulignement
	%\usepackage{soul}
	%\usepackage{ulem}

%Symbole euro
	%\usepackage{eurosym}

%Pack de polices (n'en inclure qu'un à la fois)
	%\usepackage{bookman}
	%\usepackage{charter}
	%\usepackage{newcent}
	\usepackage{lmodern}
	%\usepackage{mathpazo}
	%\usepackage{mathptmx}

%Citation d'URL
	%\usepackage{url}

%liens internes et externes
	\usepackage{hyperref}

%Citation de code
	\usepackage{verbatim}
	%\usepackage{moreverb}
	\usepackage{fancyvrb}

%Citation de code coloré
	\usepackage{listings}

%En-têtes et pieds de pages personnalisés
	\usepackage{fancyhdr}

%Insertion d'images
	\usepackage{graphicx}

%Insertion d'une image dans un paragraphe
	%\usepackage{wrapfig}

%Manipuler les couleurs et colorer du texte
	\usepackage{xcolor}

%Colorer du texte contenu dans un tableau
	%\usepackage{colortbl}

%Insertion d'expressions scientifiques
	\usepackage{amsmath}
	\usepackage{amssymb}
	\usepackage{mathrsfs}
	%\usepackage{asmthm}

%Création d'index
	%\usepackage{makeidx}

%Modification des styles de listes
	%\usepackage{enumitem}

%Modification des styles des titres (voir Original title spec with package titlesec.tex et titlespec.pdf)
	\usepackage[nobottomtitles]{titlesec}

%Highly customized stacking of objects, insets, baseline changes, etc.
	\usepackage{stackengine}

%Routines for constrained scaling and stretching of objects, relative to a reference object or in absolute terms
	\usepackage{scalerel}

%change some distances
	\usepackage{tocloft}
	\usepackage{titletoc}
%Definition of colors
\definecolor{mygreen}{HTML}{256428}
\definecolor{myblue}{HTML}{2e75b6}
\definecolor{codecolor}{HTML}{205280}

%Prevent page breaks in paragraphs
\predisplaypenalty=1000
\postdisplaypenalty=1000
\clubpenalty=1000

%Minimal space required in the bottom margin not to move the title on the next page
\renewcommand{\bottomtitlespace}{.1\textheight}

%redefinition of the title numbering
%\renewcommand{\thechapter}{Chap.\arabic{chapter}}
\renewcommand{\thesection}{\Roman{section}.}
\renewcommand{\thesubsection}{\arabic{subsection}.}
\renewcommand{\thesubsubsection}{\alph{subsubsection}.}

%Definition of the format of my titles
%\titleformat*{command}{format}
%\titleformat{command}[shape]{format}{label}{sep}{before-code}[after-code]
\titleformat{\chapter}
{\normalfont\scshape\huge\color{myblue}}{Chapitre \thechapter:}{0.5em}{}

\titleformat{\section}
{\normalfont\Large\bfseries\color{myblue}}{\thesection}{0.5em}{}

\titleformat{\subsection}
{\normalfont\large\bfseries\color{myblue}}{\hspace{15pt}\thesubsection}{0.5em}{}

\titleformat{\subsubsection}
{\normalfont\normalsize\bfseries\color{myblue}}{\hspace{30pt}\thesubsubsection}{0.5em}{}

\titleformat{\paragraph}
{\normalfont\normalsize\bfseries\color{mygreen}}{\theparagraph:}{1em}{}

\titleformat{\subparagraph}[runin]
{\normalfont\normalsize\bfseries}{\thesubparagraph}{1em}{}

%\titlespacing*{command}	{left}{before-sep}{after-sep}[right-sep]
\titlespacing*{\chapter}     {0pt}{0pt}{30pt}
\titlespacing*{\section}     {0pt}{4.5ex plus 1ex minus .2ex}{2.3ex plus .2ex}
\titlespacing*{\subsection}   {0pt}{3.25ex plus 1ex minus .2ex}{1.5ex plus .2ex}
\titlespacing*{\subsubsection}{0pt}{3.25ex plus 1ex minus .2ex}{1.5ex plus .2ex}
\titlespacing*{\paragraph}   {0pt}{3.25ex plus 1ex minus .2ex}{1em}
\titlespacing*{\subparagraph} {\parindent}{3.25ex plus 1ex minus .2ex}{1em}

%option for cover
\title{TZ20 - Rapport}
\author{PINARD Maxime - LAZARE Lucas}
\date{A2015}

%Fancy style options
\lhead{\small{TZ20 - Rapport}}
\rhead{\small{PINARD Maxime - LAZARE Lucas}}
\lfoot{\small{UTBM}}
\rfoot{\small{\today}}
\pagestyle{fancy}

%Redefine the plain page style
\fancypagestyle{plain}{
	\fancyhf{}
	\lhead{\small{TZ20 - Rapport}}
	\rhead{\small{PINARD Maxime - LAZARE Lucas}}
	\lfoot{\small{UTBM}}
	\rfoot{\small{\today}}
	\cfoot{\thepage}
}

% Table of contents to list down to 'titleimportance' 5
\setcounter{tocdepth}{5}

% Number down to 'lasttitleimportance' to list
\setcounter{secnumdepth}{10}

%link option, especialy for the table of contents
\hypersetup{
    colorlinks=true,
    linkcolor=black,
    urlcolor=blue,
    linktoc=all
}

%table of content space after sections number
%\setlength{\cftbeforetoctitleskip}{0pt}% Set space before TOC
\setlength{\cftsecnumwidth}{25pt}% Set numwidth of section
\setlength{\cftsubsecnumwidth}{25pt}% Set numwidth of subsection

%environment 'lstlisting' settings
\lstset{ %
language=C++,                                    % choose the language of the code
basicstyle=\normalsize\color{codecolor},
keywordstyle=\textit,
commentstyle=\itshape\color{gray},
identifierstyle=\textbf,
stringstyle=\color{red},
numbers=left,                                    % where to put the line-numbers
numberstyle=\scriptsize,                         % the size of the fonts that are used for the line-numbers
stepnumber=1,                                    % the step between two line-numbers. If it is 1 each line will be numbered
numbersep=7pt,                                   % how far the line-numbers are from the code
backgroundcolor=\color{yellow!3},                % choose the background color. You must add \usepackage{color}
showspaces=false,                                % show spaces adding particular underscores
showstringspaces=false,                          % underline spaces within strings
showtabs=false,                                  % show tabs within strings adding particular underscores
frame=single,                                    % adds a frame around the code
tabsize=4,                                       % sets default tabsize to 2 spaces
captionpos=b,                                    % sets the caption-position to bottom
breaklines=true,                                 % sets automatic line breaking
breakatwhitespace=true,                          % sets if automatic breaks should only happen at whitespace
emph={if, then, else, endif, for, foreach, while, do, done, switch, case, break, end, procedure, function, return},
emphstyle={\color{blue!50}\bfseries}
}

\frenchbsetup{StandardLayout=true,ReduceListSpacing=false,CompactItemize=false}

\begin{document}
	\maketitle{}
	\renewcommand{\contentsname}{Sommaire} %redefinition of the tableofcontent title
	\tableofcontents{}
	\newpage{}
	\section{Introduction} %I
		\subparagraph*{}
			Et si nous programmions nous même un serveur et son client, en partant de rien, plutôt que d'utiliser un programme préexistant? C'est la question que nous nous sommes posés lorsque nous avons décidés de commencer ce projet. En effet, bien qu'il existe déjà de nombreux programmes pouvant faire cela — comme Apache, par exemple —, nous voulions comprendre la façon dont ces programmes fonctionnent. S'il est vrai que nous aurions pu simplement lire les sources d'Apache, nous avons pensés qu'il serait plus instructif de le refaire nous même. Ce faisant, nous devions décider de:
			\begin{itemize}
				\item{} Comment organiser les fichiers sur le serveur.
				\item{} Quelles sont les tâches respectives du client et du serveur.
				\item{} Comment ceux-ci communiquerons-t-ils.
				\item{} Quelles sont les autorisations du client.
			\end{itemize}
		\subparagraph*{}
			Afin de répondre à ses questions, nous avons longuement réfléchi, ces questions étant importantes. L'organisation des fichiers, par exemple, règle pour beaucoup la façon dont le serveur fonctionne
	\section{Description des objectifs et énoncés des problèmes} %II
		\subparagraph*{}
			L'objectif du projet TSiD est de coder un serveur et un client fonctionnels en ligne de commande, cela en utilisant la bibliothèque réseau SFML pour communiquer par internet.
			Les fonctionnalités seront les suivantes:
			\begin{itemize}
				\item{} L'accès au serveur est restreint aux membres qui possèdent un compte
				\item{} Chaque membre a accès à un dossier publique ainsi qu'a un dossier privé. Seul la personne a qui appartient le dossier privé peut y accéder
				\item{} Lorsque le client liste les fichiers d'un dossier, la date de création ainsi que le nom du créateur de chaque fichier lui sont aussi indiqués
				\item{} Un utilisateur, après avoir uploadé un fichier dans l'espace publique peut le supprimer dans les 24 heures
				\item{} Le serveur enregistre des logs de connexion
				\item{} Les utilisateurs peuvent changer de mot de passe
				\item{} Un utilisateur peut ajouter une description a un fichier qu'il a uploadé
				\item{} Un utilisateur peut créer un compte pour quelqu'un (c'est le seul moyen de créer un compte)
				\item{} Le serveur est configurable (ex: autoriser ou non aux utilisateurs l'accès a leur dossier privé)
			\end{itemize}
			Aussi, plus tard peut être, le serveur ne devra pas enregistrer les mots de passe des utilisateurs mais un hash de ceux-ci.
			Les deux programmes (client et serveur) doivent être disponibles sous les deux systèmes d'exploitation: Windows et Linux
			Pour implémenter ces fonctionnalités nous auront à résoudre les problèmes suivants:
			\begin{itemize}
				\item{} Comment communiqueront le client et le serveur? Comment traiter les commandes envoyées par le client?
				\item{} Comment traiter les accès aux dossier ainsi que les restrictions d'accès aux dossier?
				\item{} Comment récupérer la liste des dossiers et fichiers présents dans un dossier?
				\item{} Comment stocker des informations à propos des dossiers et fichiers uploadés?
				\item{} Comment afficher des informations de façon ergonomique/lisible (avec des couleurs)?
				\item{} Comment envoyer des mails via un programme?
				\item{} Comment créer, gérer et utiliser la configuration du serveur?
				\item{} Comment permettre au client de télécharger / uploader un dossier complet?
			\end{itemize}
			Nous devont résoudre chaque problème sur les deux systèmes d'exploitation (Linux \& Windows)
	\section{Solutions choisies} %III
		\subsection{Client - server: communication} %1
			\subparagraph*{}
				Afin de permettre la communication entre le client et le serveur, nous avons dû mettre en place une \textit{'grammaire standard'}. Celle-ci fonctionne de la façon suivante :
				\begin{itemize}
					\item{} Le client envoie : \textit{'Je veux faire ça, ici'}
					\item{} Le serveur essaiera ensuite de remplir la commande du client. Si cela est possible, il répondra au client \textit{'D\'accord, fait le'}, ou renverra directement la réponse attendue. Sinon, il renverra la réponse apropriée (action non autorisée, une erreur est survenue, \ldots{})
					\item{} Si le client doit effectuer une action supplémentaire (ie : téléverser un fichier), il le fera.
				\end{itemize}
			\subparagraph*{}
				Pour la communication de bas niveau (envoyer et recevoir une variable), nous avons utilisés la librairie \textbf{SFML/Network}, qui fournit toutes les fonctions nécessaire à l'envoi et la réception de variables par internet. Nous utilisons nottament : \lstinline$sf::TcpSocket::send( sf::Packet& packet )$ et \lstinline$sf::TcpSocket::receive( sf::Packet& packet )$, ou sf::Packet et un flux d'entrée/sortie.
				Algorithme ci-après : le client téléverse un fichier
			\paragraph*{Pseudo-code, client:}
				\begin{lstlisting}[breaklines]
function uploadAFile( remote_working_directory: string, name_of_file_to_upload: string ): boolean
|
|	server_answer: integer
|	file: table of characters
|
|	sendToServer( upload ) //where 'upload' is associated to a command code (integer)
|	sendToServer( remote_working_directory )
|	receiveFromServer( server_answer )
|	
|	if isPositive( server_answer ) = true then
|	|	
|	|	file <- readFile( name_of_file_to_upload )
|	|	sendToServer( name_of_file_to_upload )
|	|	sendToServer( file )
|	|	return true
|	|
|	else
|	|	return false
|	endif
|	
done
\end{lstlisting}
			\paragraph*{Pseudo-code, server:}
				\begin{lstlisting}[breaklines]
procedure mainLoop()
|
|	directory: table of characters
|	command: integer
|	
|	while true do
|	|
|	|	receiveFromClient( command )
|	|	receiveFromClient( directory )
|	|
|	|	switch command
|	|	|	.
|	|	|	.
|	|	|	.
|	|	|	case upload: retrieveAFile( directory )
|	|	|	.
|	|	|	.
|	|	|	.
|	|	end
|	done
done
\end{lstlisting}
				\begin{lstlisting}[breaklines]
procedure retrieveAFile( directory: table of characters )
|
|	file: table of characters
|	file_name: table of characters
|	
|	if ClientIsAllowedToUploadThere( directory ) = true then
|	|
|	|	sendToClient( posivite_answer ) //where 'positive_answer' is associated to an answer code (integer)
|	|	receiveFromClient( file_name )
|	|	receiveFromClient( file )
|	|	write( file_name, file )
|	|	
|	else
|	|	sendToClient( negative_answer ) //where 'negative_answer' is associated to an answer code (integer)
|	endif
done
\end{lstlisting}
		\subsection{Server: Restriction d'accès} %2
			note: \textit{'./'} désigne le dossier courant et \textit{'../'} désigne le dossier parent.
			\subparagraph*{}
				L'objectif de cette fonctionnalitée est d'empêcher au client d'acceder aux dossiers qui ne sont pas dans l'architecture du serveur même, mais à ceux existant dans des dossiers parents. Il faut egalement restreindre l'accès aux dossiers personnels, pour empêcher un membre d'accéder au dossier privé d'un autre membre.
				Afin de remplir cette contrainte, nous analyserons simplement le chemin d'accès envoyé par le client.
			\subparagraph*{}
				Pour le premier point, nous supposerons que le client utilisé est bien celui que nous avons programmé, et pas un autre fait par un tiers. En particulier, les chaînes \textit{'/..'} et \textit{'../'} ne sont jamais présentes dans le chemin d'accès fourni par le client. Par  concéquant, si l'une d'entre elles est trouvée, l'action du client sera refusée, quelle qu'elle soit.
			\subparagraph*{}
				Pour la deuxième vérification, nous utiliseront l'architecture des dossiers du serveur (avec \textit{foo} et \textit{bar} des membres inscrits) :
				\begin{samepage}
					\begin{verbatim}
server execution folder/
                         — Public/
                                    — SomeFolders/
                                    — SomeFile.ext
                         — Private/
                                    — Jack/
                                           — JacksFolder/
                                           — JacksFile.ext

                                    — Robert/
                                           — RobertsFolder/
                                           — RobertsFile.ext
\end{verbatim}   

				\end{samepage}
			\subparagraph*{}
				Listage de fichiers et comportement du client pour le chemin d'accès :
				\begin{itemize}
					\item{} Le client considère qu'il accède par défaut dans \textit{'/'}, mais cela est reéinterprété par le serveur en \textit{'./Public'}.
					\item{} Si le client demande à lister les fichiers et dossiers dans \textit{'/'}, il verra les fichiers de \textit{'./Public'}, ainsi qu'un dossier supplémentaire, \textit{'Private/'}.
					\item{} Si le client demande l'accès à \textit{'/Private'}, le serveur le redirigera vers \textit{'./Private/client\_id'} silencieusement (le client affichera être dans \textit{'/Private'}).
				 \end{itemize}
			\subparagraph*{}
				Pour faire ces vérifications, il nous faut modifier le code du serveur. Aucune vérification d'accès de doit être effectuée de côté du client.
				\paragraph*{Arlgorithme : restrictions d'accès (note : client\_id est connu) :}
				\begin{lstlisting}
procedure main()
|
|	directory: table of characters
|	command: integer
|	
|	while true do
|	|
|	|	recieveFromClient( command )
|	|	receiveFromClient( directory )
|	|	
|	|	if formatPath( directory, client_id ) = true then
|	|	|
|	|	|	switch command
|	|	|	|	.
|	|	|	|	.
|	|	|	|	.
|	|	|	|	.
|	|	|	end
|	|	|
|	|	else
|	|	|	sendToClient( prohibited ) //where 'prohibited' is associated to an answer code (integer)
|	|	endif
|	done
done
\end{lstlisting}
				\begin{lstlisting}
function formatPath( directory: table of characters, client_id: table_of_characters ): boolean
|
|	if (find( directory, "/../" ) = true) or (startsby( directory, "../" ) = true) then
|	|	return false
|	endif
|
|	if endsBy( directory, "/.." ) = true then
|	|	return false
|	endif
|
|	if startsBy( directory, "/Private") = true then
|	|	insert( directory, 9, "/" + client_id ) //inserts /client_id right after /Private
|	else
|	|	directory <- "/Public" + directory
|	endif
|	directory <- "." + directory //The directory is not from the disk's root, but from the server execution folder
|	return true
|
done
\end{lstlisting}

		\subsection{Client: Affichage} %3
			\subparagraph*{}
				Afin que le membre puisse comprendre l'affichage, il faut que celui-ci soit lisible. Il aurait été par exemple très facile de faire un affichage comme celui-ci :
				\begin{Verbatim}[frame=single]
Pictures/ sample.mp3 sample.mp4 Movies/ Documents/ Private/
\end{Verbatim}
				Cepedant, c'est assez difficile à lire et à comprendre. Nous avons donc opté pour un affichage obéissant aux règles suivantes :
				\begin{itemize}
					\item{} Afficher les éléments ligne par ligne.
					\item{} Afficher en premier les dossiers (en bleu), puis les fichiers (en vert), en les séparant d'une line vide.
					\item{} Trier les fichiers et dossiers par ordre alphabetique.
					\item{} Ajouter des colonnes indiquant la date de création et le pseudo du créateur.
				\end{itemize}
				Le résultat final donne donc :
				\begin{Verbatim}[commandchars=\\\{\},codes={\catcode`$=3\catcode`^=7\catcode`_=8},frame=single]
Name         Creat. Date  Creator
\color{blue}Documents/
\color{blue}Movies/
\color{blue}Pictures/
\color{blue}Private/

\color{green}sample.mp3   Fri 06/11/15 foo
\color{green}sample.mp4   Sat 07/11/15 bar
\end{Verbatim}

			\subparagraph*{}
				Nous avons également décidé d'indiquer les pourcentages de téléchargement/téléversement avec un affichage en style \textit{'pacman'}. Afin d'être toujours lisible, l'affichage doit respecter les règles suivants :
				\begin{itemize}
					\item{} Si moins de 10 caractères sont disponibles pour l'affichage, on n'affiche que des \textit{'–'}.
					\item{} Si 11 à 34 caractères sont disponibles, le nom du fichier et le nombre d'octets transferés est affiché.
					\item{} Si 35 à 45 caractères sont disponibles, on affiche en plus le nombre total d'octets à transferer.
					\item{} Si plus de 45 caractères sont disponibles, on affiche également le pourcentage \textit{'pacmanisé'}, qui prendra au plus 1/3 des caractères disponibles.
				 \end{itemize}
		\subsection{Server: Stockage des informations sur les fichiers} %4
			\subparagraph*{}
				Il peut être interessant de stocker des informations sur les fichiers ayant été uploadés, tels que la date de création et l'auteur.
				Pour ce faire, l'architecture suivante sera utilisée :
				\begin{samepage}
					\begin{verbatim}
server execution folder/
                         — Public/
                                    — SomeFolders/
                                    — SomeFiles.ext
                         — Private/
                                    — Jack/
                                           — JacksFolder/
                                           — JacksFile.ext

                         — FilesData/
                                      — Public/
                                                 — SomeFolders/
                                                 – .SomeFolders
                                                 — .SomeFiles.ext
                                      — Private/
                                                 — Jack/
                                                        — JacksFolder/
                                                        — .JacksFolder
                                                        — .JecksFile.ext
\end{verbatim}

				\end{samepage}
				Où \textit{.Somefolders}, \textit{.Somefiles.ext}, \ldost{} contiennent les informations concernant \textit{SomeFolders/}, \textit{SomeFiles.ext}, \ldots{}
				Un point est ajouté au début du nom du fichier / dossier afin de permettre la création d'un dossier et de sa description au même endroit. Il est à noter que la crétion d'un dossier commençant par un point doit être interdite.
			\subparagraph*{}
				Ainsi, lorsque \textit{foo} souhaite uploader le fichier \textit{'file'} dans \textit{'directory'}, il suffit de :
				\begin{itemize}
					\item{} Écrire \textit{'file'} dans \textit{'directory/'}
					\item{} Écrire la date et \textit{'foo'} dans \textit{'./Filesdata/directory/.file'}
				\end{itemize}
				Algorithme utilisé, en modifiant la procédure retrieveAFile (pas de changement dans le client) :
			\paragraph*{Algorithme, serveur:}
				\begin{lstlisting}
procedure retrieveAFile( directory: table of characters, client_id: table of characters )
|
|	file: table of characters
|	file_name: table of characters
|	
|	if ClientIsAllowedToUploadThere( directory ) = true then
|	|
|	|	sendToClient( posivite_answer ) //where 'positive_answer' is associated to an answer code (integer)
|	|	receiveFromClient( file_name )
|	|	receiveFromClient( file )
|	|	write( file_name, file )
|	|	writeFileInformations( directory, file_name, client_id )
|	|	
|	else
|	|	sendToClient( negative_answer ) //where 'negative_answer' is associated to an answer code (integer)
|	endif
done
\end{lstlisting}
				\begin{lstlisting}
procedure writeFileInformations( directory: table of characters, file_name: table of characters, client_id: table of characters )
|
|	date: table of characters
|	
|	date <- retrieveDate()
|	makeDirectory( directory )
|	write( date + NEWLINE + client_id, "./FilesData" + directory + "." + file_name ) //writes the file's info in FilesData/directory/.filname
|	
done
\end{lstlisting}
		\subsection{Client: upload / download récursif} %5
			\subparagraph*{}
				Afin de permettre au client de télécharger et de téléverser un dossier, nous avons choisi une approche récursive, fonctionnant de la façon suivante :
				\begin{itemize}
					\item[\textbf{1 }]{} Ouvrir le dossier et lire le premier élément qu'il contient et aller à l'étape suivante.
					\item[\textbf{2a}]{} Si c'est un dossier, recommencer l'étape 1 avec ce nouveau dossier, puis aller à l'étape 3.
					\item[\textbf{2b}]{} Sinon, le télécharger/verser, puis aller à l'étape suivante.
					\item[\textbf{3 }]{} Retourner à l'étape 2 avec l'élément suivant.
				\end{itemize}
				Algorithme du télécharement récursif (pas de changements au serveur) :
			\paragraph*{Algorithme, client:}
				\begin{lstlisting}
function recursiveDownload( remote_working_directory: table of characters ): boolean
|
|	successful: boolean
|	server_answer: integer
|	i: integer
|	file_list: table of table of characters
|
|	sendToServer( listFiles )		//Where 'listFiles' is associated to a command code
|	sendToServer( remote_working_directory )
|	receiveFromServer( server_answer )
|
|	if server_answer = negative_answer then	//Where 'negative_answer is associated to some answer codes
|	|	return false
|	endif
|
|	successful <- true	
|	
|	receiveFromServer( file_list )		//file_list now contains a list of the files and folders that are within 'remote_working_directory'
|
|	for i in [1..file_list.size()]		//from 1 to the number of files/folder within 'file_list'
|	|
|	|	if endsBy( file_list[i], "/" ) then	//if 'file_list[i]' ends by '/', it is a folder
|	|	|	recursiveDownload( remote_working_directory + '/' + file_list[i] )
|	|	|
|	|	else
|	|	|	success <- download( remote_working_directory + '/' + file_list[i] ) and success	//'success' is true if the download was successful AND if it was true before
|	|	endif
|	done
|
|	return success
|
done
\end{lstlisting}
		\subsection{Serveur : Configuration} %6
			\subparagraph*{}
				We want to have a configuration for the server. There is a list of the configurable elements for now:
				\begin{itemize}
					\item{} Generate or not the server folders at next startup
					\item{} Create a new user or not at next startup
					\item{} Allow members to invite another member (basically: create a new account) or not
					\item{} Allow members to write in their private folders or not
					\item{} Allow members to download from their private folders or not
				\end{itemize}
			\subparagraph*{}
				To do so, we decided to create an object Config, having the following private variables:
				\begin{itemize}
					\item{} user\_creation\_allowed
					\item{} private\_folder\_writing\_allowed
					\item{} private\_folder\_reading\_allowed
				\end{itemize}
				all booleans\\
				This object will be in read-only.\\
				The first two setting do not appear in the object, as they are used only at startup.\\
				A pointer to this object will be passed to each client's thread.
		\subsection{Server, client: Message d'accueil} %7
			\subparagraph*{}
				The objective is to allow the the server owner to write a user-adjustable welcome message for all users.
				In this way we wanted to implement some variables in the welcome message, so we used the syntax \textit{'\$[variable\_name]'}, the available variables are:
				\begin{itemize}
					\item{} \textbf{user} the user name
					\item{} \textbf{date} the date formated \textit{'dd/mm/yy'}
					\item{} \textbf{day} the day formated \textit{'Mon,Tue,Wed,Thu\ldots'}
					\item{} \textbf{hour} the hour formated \textit{'hh:mm'}
					\item{} \textbf{color} to set the text color from the variable to the next color variable
				\end{itemize}
				where \textbf{color} can be blue, green, cyan, red, magenta, yellow, white.
				To put the \textit{'\$'} symbol simply put \textit{'\$\$'}
			\paragraph*{Pseudo-code, server:}
				\begin{lstlisting}
procedure mainLoop()
|
|	formatedWelcomeMessage( message, client_id )
|	SendToClient( message )
|	
|	while ClientConnected() do
|	|	.
|	|	.
|	|	.
|	done
done
\end{lstlisting}
				\begin{lstlisting}
Procedure formatedWelcomeMessage(message: table of characters, client_id: table of characters)
|
|	read( "WelcomeMessage.txt", message )
|	
|	foreach $[command] in message do
|	|	
|	|	switch command
|	|	|	
|	|	|	case user: remplace( $[command], client_id )
|	|	|	
|	|	|	case date: remplace( $[command], getDate() )
|	|	|	           //formated 'dd/mm/yy'
|	|	|	
|	|	|	case day: remplace( $[command], getDay() )
|	|	|	           //formated 'Mon,Tue,Wed,Thu...'
|	|	|	
|	|	|	case hour: remplace( $[command], getHour() )
|	|	|	           //formated 'hh:mm'
|	|	|	
|	|	|	default: //let the client interpret $$ and $[color]
|	|	end
|	done
done
\end{lstlisting}
			\paragraph*{Pseudo-code, client:}
				\begin{lstlisting}
Procedure main()
|	
|	connectToServer()
|	InterpretWelcomeMessage()
|	
|	while ConnectedToServer() do
|	|	.
|	|	.
|	|	.
|	done
done
\end{lstlisting}
				\begin{lstlisting}
Procedure InterpretWelcomeMessage()
|	
|	ReceiveFromServer(message)
|	
|	while( not_the_end_of_the_message )
|	|	
|	|	Print( GetTextUntilSymbol(message, '$') )
|	|	
|	|	switch( GetNextChar(message) ) //switch the char rigth after $
|	|	|	
|	|	|	case '$': Print('$') //if there is $$ in the message, print $
|	|	|	
|	|	|	case '[': //there is $[color] in the message, change the color
|	|	|	|	
|	|	|	|	color <- GetTextUntilChar(']') //color take the value in $[color]
|	|	|	|	
|	|	|	|	if( IsAPrintableColor( color ) ) then
|	|	|	|	|
|	|	|	|	|	SetTextColor( color )
|	|	|	|	|
|	|	|	|	endif
|	|	|	|	
|	|	|	break
|	|	end
|	done
done
\end{lstlisting}

	\section{Conclusion} %IV
		\subparagraph*{}
			Ce projet fut très instructif et intéressant. La dernière version du programme pour l'UV de TZ20 et la V1 Ranitomeya reticulata, elle ne contient pas tout ce que nous aurions voulu implémenter mais les fonctionnalités les plus importantes ont été implémentées. Nous avons la volonté de continuer à travailler sur ce programme. La version actuelle peut être utilisée avec des personnes en lesquelles vous avez confiance, mais la sécurité n’ayant pas été testée nous ne recommandons pas son utilisation à plus grande échelle.
		\subsection{Améliorations possibles} %1
			\subparagraph*{}
				Certaines fonctionnalités mériteraient d'être plus développées mais ne l'ont pas été par manque de temps, par exemple:
				\begin{itemize}
					\item{} \textbf{Suppression de fichiers}\\
						Les utilisateurs peuvent uploader des fichiers et dossier mais ils ne peuvent pas les supprimer. Nous avions pensé à un système de vote ou les utilisateurs auraient pu voter pour ou contre la suppression d'un fichier.
					\item{} \textbf{Mots de passe}\\
						Les mots de passe sont enregistrés tel-quel sur le serveur ce qui n'est pas assez sécurisé. Il serait préférable de sauvegarder des hash de ces derniers.
					\item{} \textbf{Mails}\\
						Nous aurions voulu implémenter un système de mails qui aurait pu être utilisé pour:
						\begin{itemize}
							\item{} Inviter des nouveaux utilisateurs
							\item{} Prévenir les membres lorsqu'une adresse IP inconnue essaye d'accèder a leur compte
							\item{} Informer un membres lorsqu'un vote a été lancer pour la suppression d'un fichier qu'il a uploadé
						\end{itemize}
					\item{} \textbf{Filtre IP}\\
						Dans le but d'aider les utilisateurs à sécuriser leur compte, nous voudrions ajouter une vérification de l'adresse IP à la connexion. Toute IP non autorisée se verra ainsi refuser la connexion au compte. Un mail pourra être envoyé à l'utilisateur pour ajouter l'IP à la liste des adresses autorisées. L'utilisateur aurait la possibilité de désactiver ce service.
					\item{} \textbf{Admin thread}\\
						There is no way to delete a user, a file, or to change the server configuration easily, which is not nice. The problem could be solved by adding an admin console, were you could enter some commands
					\item{} \textbf{Limit private folder space usage}\\
						The goal of this server is mainly to use the public folder, so it could be nice to set a limit to the private folder, let's say 10Gio, configurable via the server's configuration
				\end{itemize}
		\subsection{Connaissances acquises} %2
			\subparagraph*{}
				What did we learn while doing this project?
				\begin{itemize}
					\item{} \textbf{Teamwork}\\
						We learned to work as a team, in a different way we did in others UV with a team presentation, such as LE03
					\item{} \textbf{Work organization}\\
						As we were working in autonomy, we had to organize us in order to meet the deadlines. We had to organize both the working order (what to do), and timing (when to do it). That may be very useful in our professional life.
					\item{} \textbf{GitHub}\\
						GitHub is a powerful tool of version gesture that helps developpers to work together on a project. This could help us as well in our active life, as we may work as developpers, and our company will probably use a similar tool
					\item{} \textbf{Strings manipulation}\\
						We learned a lot about strings [in C++], and that knowledge can easily be transfered most of others programming languages
					\item{} \textbf{Cmake}\\
						Finally, we learned about CMake, which is a powerful tool to generate MakeFiles.
				\end{itemize}
\end{document}
